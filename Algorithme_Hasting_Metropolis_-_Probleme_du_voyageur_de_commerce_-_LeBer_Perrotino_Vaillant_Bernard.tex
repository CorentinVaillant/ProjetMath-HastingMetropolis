\documentclass{article}

% Gestion des polices en fonction de l'encodage (PDFLaTeX, XeLaTeX, LuaTeX ou autre) :
\usepackage{ifxetex}
\usepackage{mathtools}
\usepackage{amssymb}
\usepackage{ifluatex}
\ifxetex
  % XeTeX est utilisé :
  \usepackage{fontspec}
\else
  \ifluatex
    % LuaTeX est utilisé :
    \usepackage{fontspec}
  \else
    % pdfLaTeX ou autre moteur est utilisé :
    \usepackage[T1]{fontenc}
    \usepackage[utf8]{inputenc}
    \usepackage{lmodern}
  \fi
\fi

% Autres packages :
\usepackage[french]{babel}
\usepackage{mdframed}
\usepackage{enumitem}
\usepackage{hyperref}
\usepackage{amsmath}
\usepackage{amsfonts}
\usepackage{stmaryrd}
\usepackage[explicit]{titlesec}
\usepackage{framed}
\usepackage{amsthm}
\usepackage{xcolor}
\usepackage{tcolorbox}
% Package permettant de générer des graphes :
\usepackage{tikz}
\usetikzlibrary{arrows}
\usetikzlibrary{arrows.meta}
\usepackage{pgfplots}
\pgfplotsset{compat=1.18} % Préciser la version de pgfplots afin qu'il soit compatible avec le type de cocument (ici "article").

\sloppy % Pour que les URLs ne dépassent pas des marges.

\title{Algorithme d'Hasting Metropolis - Problème du voyageur de commerce}
\author{PEROTTINO Tony, VAILLANT Corentin,\\ LE BER Tom, BERNARD Léo}

% Enable SageTeX to run SageMath code right inside this LaTeX file.
% http://doc.sagemath.org/html/en/tutorial/sagetex.html
% \usepackage{sagetex}

% Enable PythonTeX to run Python – https://ctan.org/pkg/pythontex
% \usepackage{pythontex}

\begin{document}
\maketitle

\newpage
\tableofcontents
\newpage

\section*{Préambule}

Ce rapport s'inscrit dans le cadre de la matière "Projet de Mathématiques" de l'université de Toulouse (Paul Sabatier). \\
L'objectif de ce rapport est de présenter et de prouver l'algorithme d'Hastings-Metropolis. À la fois d'expliquer en détail son fonctionnement, mais aussi de comprendre l'intérêt qu'il présente dans les sciences et en quoi il prend sa valeur. \\
Pour illustrer l'algorithme, nous nous intéresserons au problème du voyageur de commerce et étudierons sa complexité.


\section{Introduction}

Afin de comprendre en profondeur ce sujet, il est nécessaire de s'en faire une idée globale, même naïve, pour pouvoir suivre convenablement la ligne directrice de notre discours.

\subsection{Informations générales et description historique}

L'algorithme d'Hastings-Metropolis consiste en une méthode d'échantillonnage stochastique permettant, à partir d'une distribution de probabilité donnée, de pouvoir décrire son comportement et d'obtenir des statistiques dessus. Cet algorithme prend sa valeur quand la distribution est difficile à analyser (systèmes multidimensionnels, par exemple). Cette méthode est marquante, car elle a été conceptualisée tôt dans l'histoire de l'informatique et a mis des décennies avant d'être prouvée et expliquée entièrement. \\
C'est en 1949 que l'écriture de l'algorithme a été publiée dans un article de Nicholas Metropolis et Stanisław Ulam. La paternité de l'algorithme est soumise à débat, car l'algorithme s'inscrit sous le nom de son chef de projet (Metropolis), alors que l'équipe composée de Nicholas Metropolis, Arianna et Marshall Rosenbluth, Augusta et Edward Teller a contribué à cette méthode. Ils étudiaient alors plus particulièrement le cas de la distribution de Boltzmann, une des distributions les plus utilisées en physique statistique, dans des travaux datant de 1953. \\
Cela illustre une dynamique fréquente dans les sciences, où le mérite est disproportionnellement attribué à une personne alors qu'il s'agit des efforts de toute une équipe. En particulier, Arianna Rosenbluth était considérée comme brillante par le monde scientifique. \\

\url{https://fr.wikipedia.org/wiki/Algorithme_de_Metropolis-Hastings#cite_note-2}

\url{https://www.radcliffe.harvard.edu/news-and-ideas/flash-of-genius} \\

En 1970, W. K. Hastings (1930-2016) a étendu l'algorithme au cas d'une distribution quelconque, et c'est cette version généralisée qui est connue sous le nom d'algorithme de Metropolis-Hastings. Cette extension a eu de nombreuses applications dans divers domaines scientifiques, comme en statistique bayésienne (espaces complexes multidimensionnels), en biologie computationnelle (analyse des séquences génétiques), en économie et en finance (modèles stochastiques MCMC en général), etc. \\

Quant à lui, le problème du voyageur de commerce (dit "TSP", comme Travelling Salesman Problem en anglais) est un problème classique et bien connu pour être un problème NP-difficile et NP-complet, ce qui signifie qu'il est extrêmement difficile de trouver une solution optimale en temps polynomial, et il est peu probable que des algorithmes en temps polynomial existent pour résoudre ce problème de manière exacte. \\
L'origine du problème est assez incertaine : il a été formulé pour la première fois vers 1850 dans un manuel d'un commerçant voyageant en Suisse et en Allemagne. Ce n'est que dans les années 1930 que le problème fut énoncé d'abord comme un casse-tête (par William Rowan et Thomas Kirkman), puis étudié (par, entre autres, Thomas Kirkman, Jillian Beardwood, J. H. Halton et John Hammersley). \\ 
Le problème consiste à déterminer le chemin le plus court passant par tous les points d'un graphe une seule fois chacun, en terminant par le point de départ (recherche d'un cycle hamiltonien le plus court). Les distances peuvent être dites symétriques ou asymétriques, c'est-à-dire que la distance entre eux varie en fonction de la direction du déplacement. On peut illustrer ce problème grâce à un voyageur de commerce devant vendre ses produits dans chacune des villes en un minimum de temps. Ce problème peut donc être naturellement représenté par un graphe.\\

% Insérer une image !
\url{https://fr.wikipedia.org/wiki/Probl\%C3\%A8me_du_voyageur_de_commerce}

\url{https://www.cambridge.org/core/journals/mathematical-proceedings-of-the-cambridge-philosophical-society/article/abs/shortest-path-through-many-points/F1C28B5730B94887F4659FCBF8A1F2BB} % Document de Thomas Kirkman, Jillian Beardwood, J.H. Halton ou John Hammersley.

\subsection{Pourquoi HM permet-il de résoudre le problème du voyageur de commerce ?}

Le principe mathématique derrière HM repose sur la construction dynamique d'une chaîne de Markov, qui n'est pas connue au préalable mais se développe au fil des itérations. À mesure que l'algorithme progresse, le comportement de cette chaîne converge vers la distribution cible, permettant d'obtenir un échantillon fiable par rapport aux états de la distribution. \\
Quant au problème du voyageur de commerce, il peut être représenté par un graphe orienté ayant un sommet pour chaque ville et une arête pour chaque temps de trajet. Nous verrons par la suite qu'une chaîne de Markov peut être associée à un graphe orienté, c'est-à-dire que le problème du voyageur de commerce est résoluble par HM. \\

De plus, nous avons précisé au préalable que le voyageur de commerce était NP-difficile et NP-complet. En conséquence, l'objectif de HM n'est pas d'apporter la réponse exacte au problème, mais une approximation fidèle de la solution, ce qui peut sembler contre-intuitif. Cette approximation est la caractéristique principale des méthodes MCMC que nous aborderons dans leur partie dédiée. \\
Plus généralement, il faut donc comprendre que HM est un outil adapté à la résolution de problèmes ayant un trop grand nombre de possibilités pour être explorées toutes dans un temps raisonnable. C'est pour cela que la résolution de ces problèmes se fait par l'approximation de la bonne solution (en étudiant donc des échantillons de toutes les possibilités). \\
Cette nuance est très importante, car elle révèle en quoi HM est un outil sophistiqué. \\ 

\subsection{Objectif de ce rapport}

Il est donc compréhensible que, pour prouver HM, nous allons devoir tout d'abord comprendre le fonctionnement des chaînes de Markov et en expliciter les définitions fondamentales, puis leur comportement sur le temps long, pour pouvoir ensuite construire la preuve mathématique derrière HM et déterminer les raisons de son fonctionnement. Nous étudierons à la fois cette preuve et ferons un détour par les méthodes MCMC (Monte-Carlo), dont HM est issu. Nous finirons par conclure ce rapport avec une implémentation personnelle de HM pour illustrer nos propos.



\newpage
% Questions pour le prof :
%   (1) Doit-t'on définir la notion de lois ?
%   (2) Doit-on séparer la définition de la propriété de Markov de celle d'une chaîne de Markov ?
%   (3) Doit-on définir la notion de processus stochastique ? Et la notion de temps discret/continu ?
%   (4) Pour la première sous-section, est-ce "Définitions" ou "Définitions fondamentales" ?
%   (5) Pour les propriétés fondamentales, la sous-section "Matrice de transistion" a été retirée, on ne la verra donc plus dans la légende, est-ce un problème ?
%   (6) Est-ce grave / à éviter que les définitions soient coupées par un changement de page ?

% Où mettre ce bloc ?
%   Pour tout couple $(i,j) \in E \times E$, on a :
%   \[
%   \mathcal{P}(j \mid i) = \mathbb{P}(X_{n+1}=j \mid X_n=i) \in [0,1].
%   \]
% 
%   Où $\mathcal{P}(j \mid i)$ correspond à la probabilité de passer dans l'état $j$ sachant $i$. Autrement dit, $\mathcal{P}(j \mid i)$ correspond à la probalilité de passer dans l'état $j$ en partant de l'état $i$.

% Objets à remplacer :
%   (1) /!\ Les "alpha" et "beta" par des "i" et des "j" lorsque l'on parle d'états quelconques (si on est dans des exemples avec des états nommés, autant les laisser, ça sera les noms "typiques" / que l'on utilisera pour TOUS les exemples).
%   (2) La définition d'états récurrents positifs.

\section{Chaînes de Markov à états finis}

\subsection{Définitions fondamentales}

% TODO = Parler des définitions et surtout placer le contexte, j'entends par là :
% - [X] En mathématiques, une chaîne de Markov est un processus de Markov à temps discret, ou à temps continu et à espace d'états discret. Un processus de Markov est un processus stochastique possédant la propriété de Markov : l'information utile pour la prédiction du futur est entièrement contenue dans l'état présent du processus et n'est pas dépendante des états antérieurs (le système n'a pas de « mémoire »).
% - [X] Dixit Wikipedia "\url{https://fr.wikipedia.org/wiki/Chaîne_de_Markov}", j'entends par là définir "temps discret/continu", "processus stochastique", etc.

\begin{tcolorbox}[colback=white,colframe=blue!80!black,title=Temps Discret et Temps Continu]
Soit $T$ un ensemble d'indices numérotées représentant le temps. \\

Il existe deux types de modélisation temporelle :
\begin{enumerate}[leftmargin=5em, label=(\arabic*)]
  \item Un processus \textbf{à temps discret} signifie que l'on considère les valeurs de la modélisation comme espacées régulièrement dans le temps.
  On peut prendre des ensembles dénombrables comme $T = \mathbb{N}$ ou $T = \mathbb{Z}$ où chaque instant est distinct.
  \item Un processus \textbf{à temps continu} signifie que $T$ n'est pas dénombrable. Il existe toujours un temps intermédiaire entre deux indices de $T$.
  Cela signifie que le processus évolue en tout instant dans un continuum temporel.
\end{enumerate}
\end{tcolorbox}

Dans ce rapport nous nous interesserons uniquement au temps discret, pour pouvoir modéliser le problème du voyageur de commerce.
De plus, cette distinction joue un rôle fondamental dans la classification et l'analyse des \textbf{processus stochastiques} et n'implique pas les mêmes théorèmes. \\

\begin{tcolorbox}[colback=white,colframe=blue!80!black,title=Processus Stochastique]
Soit $(\Omega, \mathcal{F}, \mathbb{P})$ un espace de probabilité.

Soit $T$ un ensemble d'indices discret ou continu (souvent $T = \mathbb{N}$ ou $\mathbb{R}_+$). \\

Un \textbf{processus stochastique} est une famille de variables aléatoires $\{X_t\}_{t \in T}$ définies sur $(\Omega, \mathcal{F}, \mathbb{P})$ et à valeurs dans un espace d'états $E$ (appelé espace d'états du processus).
\end{tcolorbox}

Un processus stochastique permet de modéliser un système évoluant de manière aléatoire en fonction du temps.

Différents types de processus peuvent être étudiés en fonction des propriétés de dépendance temporelle et de la nature de l'espace d'états $E$.
Au cours de ce rapport, nous nous concentrerons l'un des principaux processus stochastiques à temps discret et à espace d'états fini ; les \textbf{chaînes de Markov}. \\

\begin{tcolorbox}[colback=white,colframe=red!80!black,title=Chaîne de Markov]
Soit $E$ un ensemble fini ou dénombrable.

Soit $(X_n)_{n \in \mathbb{N}}$ une suite de variables aléatoires à valeurs dans l'espace d'états $E$. \\

$(X_n)_{n \in \mathbb{N}}$ est appellée \textbf{chaîne de Markov} si et seulement si :
\begin{enumerate}[leftmargin=5em, label=(\arabic*)]
  \item Sa loi de probabilité initiale $X_0$ est bien définie.
  \item Elle respecte la \textbf{propriété de Markov}, telle que :
\end{enumerate}
\[
\forall n \geq 0, \quad \exists x_1, x_2, \dots, x_{n+1} \in E,
\]
\[
\mathbb{P}(X_{n+1} = x_{n+1} \mid X_1 = x_1, \dots, X_n = x_n) = \mathbb{P}(X_{n+1} = x_{n+1} \mid X_n = x_n)
\]
\end{tcolorbox}

Une chaîne de Markov est un processus de Markov à temps discret ou à temps continu et à espace d'états discret. Un processus de Markov est un processus stochastique possédant la propriété de Markov : l'information utile pour la prédiction du futur est entièrement contenue dans l'état présent du processus et n'est pas dépendante des ses états antérieurs.

Autrement dit, la loi de probabilité $\mathbb{P}$ régissant la transition de l'état présent $X_n$ vers l'état futur $X_{n+1}$ dépend uniquement du dernier terme $X_n$, et reste totalement indépendante des tous ses états antérieurs $\{X_0, X_1, \dots, X_{n-1}\}$.

Cette propriété, que l'on peut qualifier de « sans mémoire » ou de propriété de Markov, constitue la caractéristique fondamentale de ces processus stochastiques.

\begin{tcolorbox}[colback=white,colframe=blue!80!black,title=Chaine de Markov Homogène]
Un chaine de Markov est dite homogène quand $\forall n \in \mathbb{N}$ :
\[
  \mathbb{P}[X_{n+1} = j \mid X_n = i] = \mathbb{P}[X_1 = j \mid X_0 = i]
\]
\end{tcolorbox}

En somme, une chaîne de Markov homogène ne dépend pas des états précédents (propriété de Markov) et garantit que son comportement reste inchangé au fil du temps, c'est-à-dire que les probabilités de transition restent constantes quelque doit $t \in T$ (homogénéité de la chaine de Markov). \\
Dans ce rapport, toutes les chaînes de Markov seront considérées comme homogènes.

\subsection{Propriétés fondamentales}

% TODO
% Propriétés à parler : 
% - [X] Matricer de transition
% - [X] Lien avec les graphes orientés
% - [X] Propriétés en n-pas

\subsubsection{Matrice de Transition}

\begin{tcolorbox}[colback=white,colframe=blue!80!black,title=Matrice stochastique par lignes/par colonnes]

  Une matrice \textbf{stochastique par lignes} (appelée aussi \textbf{stochastique à droite}) est une matrice dont la somme des probabilités de ses lignes vaut 1 chacune et toutes ses probabilités sont positives :
  \[
  \forall i, \quad \sum_{j=1}^{N} P_{ij} = 1 \quad \text{et} \quad P_{ij} \geq 0.
  \]
  Respectivement, une matrice est \textbf{stochastique par colonnes} (dite aussi \textbf{stochastique à gauche}) lorsque :
  \[
  \forall j, \quad \sum_{i=1}^{N} P_{ij} = 1 \quad \text{et} \quad P_{ij} \geq 0.
  \]
  Une matrice est dite \textbf{bistochastique} lorsqu'elle est à la fois stochastique par lignes et par colonnes.
\end{tcolorbox}

\begin{tcolorbox}[colback=white,colframe=red!80!black,title=Matrice de Transition]
Soit $(X_n)_{n \in \mathbb{N}}$ une chaîne de Markov homogène à valeurs dans un ensemble fini ou dénombrable $E$.

Soit la famille de nombres réels $(X_n)_{n \in \mathbb{N}}$ permettant de passer d'un état à une autre, définie comme :
\[
\forall n \geq 0,\; \forall i,j \in E,\quad P_{ij} = \mathbb{P}(X_{n+1} = j \mid X_n = i).
\]

La \textbf{matrice de transition} $P$ regroupe ces probabilités tel que chaque élément $P_{ij}$ représente la probabilité de passer de l'état $i$ à l'état $j$ en une seule étape.
Lorsque l'ensemble des états $E$ est discret, par exemple $E = \{1, 2, \dots, N\}$, la matrice s'exprime sous la forme suivante :
\[
P = \begin{bmatrix}
P_{11} & P_{12} & \cdots & P_{1N}\\[1mm]
P_{21} & P_{22} & \cdots & P_{2N}\\[1mm]
\vdots & \vdots & \ddots & \vdots\\[1mm]
P_{N1} & P_{N2} & \cdots & P_{NN}
\end{bmatrix}
\]

\end{tcolorbox}

Dans le cadre des chaines de Markov homogènes, toutes les matrices de transition $P$ sont stochastiques par lignes car les probabilités de transition depuis chaque état sont normalisées. C'est-à-dire que la somme des probabilités pour passer d'un état à un autre vaut 1. \\

De manière analogue, si l'espace des états est infini dénombrable (par exemple $E = \{1, 2, 3, \dots\}$), on indexe les états de la même façon et la condition
\[
\forall i \in E, \quad \sum_{j \in E} P_{ij} = 1,
\]
reste valable. \\

\subsubsection{Représentation sous forme de Graphe Orienté}
\label{subsubsec: Représentation sous forme de graphe orienté}

\begin{tcolorbox}[colback=white,colframe=red!80!black,title=Graphe Orienté d'une Chaîne de Markov]
Une chaîne de Markov homogène peut toujours être représentée sous forme d'un graphe orienté $ G = (V, A) $, où :
\begin{enumerate}[leftmargin=5em, label=(\arabic*)]
    \item $ V $ est l'ensemble des sommets, correspondant aux états de l'espace d'états $ E $.
    \item $ A $ est l'ensemble des arcs, où chaque arc possède une pondération correspondant à la probabilité de transition $P_{i,j}$. Un arc de $i$ vers $j$ est représenté que si $P_{i,j} > 0$.
\end{enumerate}
\end{tcolorbox}

Représenter graphiquement une chaîne de Markov homogène permet de clarifier visuellement les différentes dynamiques de transitions entre chaque état du système et de comprendre la structure du processus stochastique. \\

Considérons par exemple une chaîne de Markov ayant trois états $\alpha, \beta, \gamma$, ordonnés respectivement, et dont la matrice de transition est donnée par :

\begin{center}
$
P = \begin{bmatrix}
0.1 & 0.4 & 0.5 \\
0.6 & 0.4 & 0 \\
0.1 & 0.1 & 0.8 \\
\end{bmatrix}
$
\end{center}

Dans cette configuration, la probabilité de transition de l'état $\alpha$ vers l'état $ \beta $ est donnée par :
\[
P_{\alpha,\beta} = P_{1,2} = \mathbb{P}(X_{n+1} = \beta \mid X_n = \alpha) = 0.4.
\]

% Je viens de me rendre compte d'un sous-entendu terrible !
% Dans mes sous-parties, à chaque fois que je parle d'un état i ou j, je parle en réalité des états d'INDICE i et j.
% Sachant que l'on a utilisé cette écriture (voir phrase du dessus), pensez-vous vraiment que ce que j'ai écrit est faux ?
% Envoyez un message sur le groupe Discord quand vous verrez ce message.

Nous obtenons ainsi le graphe suivant :

\begin{center}
\begin{tikzpicture}[
    >={Latex[length=3mm, width=2mm]},
    node distance=2cm,
    state/.style={circle, draw, minimum size=0.5cm, font=\large},
    every edge/.append style={draw, -latex, font=\small}
]

% Nodes :
\node[state] (A) at (0, 0) {$\alpha$};
\node[state] (B) at (2, 3) {$\beta$};
\node[state] (C) at (4, 0) {$\gamma$};

% Edges :
\draw[->] (A) edge[loop left, out=150, in=210, looseness=8] node[left] {0.1} (A);
\draw[->] (A) edge[bend right=20] node[left] {0.4} (B);
\draw[->] (A) edge[bend right=16] node[below] {0.5} (C);

\draw[->] (B) edge[bend right=20] node[left] {0.6} (A);
\draw[->] (B) edge[loop above, out=60, in=120, looseness=8] node[above] {0.4} (B);

\draw[->] (C) edge[bend right=16] node[below] {0.1} (A);
\draw[->] (C) edge[bend right=20] node[right] {0.1} (B);
\draw[->] (C) edge[loop right, out=330, in=30, looseness=8] node[right] {0.8} (C);

\end{tikzpicture}
\end{center}

\subsubsection{Matrice de transition en \texorpdfstring{$k$}{k}-pas} % Pour que "hyperref" supporte l'expression mathématique "$k$" dans la chaîne de caractère générant un des signets du PDF.

\begin{tcolorbox}[colback=white,colframe=red!80!black,title=Matrice de transition en $n$ transitions]
Soit $(X_n)_{n \in \mathbb{N}}$ une chaine de Markov homogène d'espace d'états $E$ fini ou dénombrable tel que $\left\{1,2,\ldots,N\right\}$. On note $P$ sa matrice de transition. \\
Pour tout état $E_i$ et $E_j$ et pour tout entier naturel $k \geq 1$, le coefficient de la ligne $i$ et de la colonne $j$ de \textbf{la matrice $P^k$ est la probabilité de passer de l'état $E_i$ à celui $E_j$ en $k$ transitions}.
\end{tcolorbox}

Les chaînes de Markov homogènes permettent de déterminer l'état d'un système après $k$ transitions, c'est-à-dire au bout du $k$-ième mouvement dans la chaine, en élevant la matrice de transition à la puissance $k$. \\
Plus formellement, on a :
\[
P_{i,j}^{(k)} = \mathbb{P}(X_k = j \mid X_0 = i).
\]
Ce résultat vient de la propriété de Markov, qui indique que la probabilité de transition d'un état $i$ à un état $j$ dépend uniquement du dernier pas réalisé, et non de tous les précédents.

\begin{tcolorbox}[colback=white,colframe=green!80!black,title=Démonstration]
La démonstration de cette propriété passe par celle de l'\textbf{équation de Chapman-Kolmogorov}, telle que $\forall i,j \in E$ et $\forall n,m \in \mathbb{N}$ :
\[
\mathbb{P}(X_{n+m} = j \mid X_0 = i) = \sum_{k \in E} \mathbb{P}(X_{m} = j \mid X_0 = k) \cdot \mathbb{P}(X_n = k \mid X_0 = i).
\]

Ce qui équivaut en termes matriciels :
\[
P_{i,j}^{(m+n)} = \sum_{k \in E} P_{i,k}^{(m)} \cdot P_{k,j}^{(n)}.
\]

Cette relation peut s'interpréter en disant que pour passer de $i$ à $j$ en $n+m$ étapes il a fallu en $n$ étapes aller de $i$ à un certain $k$ puis en $m$ étapes aller de $k$ à $j$. \\

On reconnait alors l'expression de l'associativité du produit matriciel tel que :
\[
P^{n+m} = P^{n}P^{m}
\]

\end{tcolorbox}

Par exemple, si nous reprenons la matrice $P$ donnée dans la sous-section ~\ref{subsubsec: Représentation sous forme de graphe orienté} ci-dessus, la matrice de transition en $5$ étapes, nommée $P^{5}$, représente l'ensemble des probabilités permettant de passer de chaque état $i$ à un état $j$ au bout d'exactement $5$ étapes :
\begin{center}
$
P^{5} = \begin{bmatrix}
0,22350 & 0,24385 & 0,53265 \\
0,24150 & 0,26140 & 0,49710 \\
0,20595 & 0,22252 & 0,57153 \\
\end{bmatrix}
$
\end{center}
Toujours en reprennant notre exemple, la probabilité de passer de l'état $\alpha$ à l'état $\beta$ au bout d'exactement $5$ étapes est donc de :
\[
P_{\alpha,\beta}^{5} = \mathcal{P}_\alpha(\beta)^{5} = 24,385\%
\]

\begin{tcolorbox}[colback=white,colframe=yellow!80!black,title=Box a supprimer/modifier (backup)]
Ce qui est à l'intérieur de cette box est la version précédente de la partie, comme j'ai pas mal modifié de choses je la laisse en backup - Tony \\

Dans une chaîne de markov, la matrice de transition en un pas, que nous avons noté $P$, décrit pour tout couple d'états $i$,$j \in \left\{ 1, 2 , \ldots, N \right\}$ la probabilité de passer du premier état au deuxième via la valeur de l'élément $P_{i,j}$. Chaque élément $P_{i,j}$ représente donc la probalitité de passer de l'état $i$ à l'état $j$ en une seule étape, ce qui se traduit mathématiquement par :
\[
\mathbb{P}(X_1 = j \mid X_0 = i)
\]
Où $X_t$ désigne l'état de la chaîne à l'instant $t$. \\

Cependant, il peut être nécessaire de chercher à déterminer la probabilité de passer de l'état $i$ à l'état $j$ après exactement $k$ étapes. Pour cela, on utilise la matrice de transition en $k$-pas, notée $P^k$, pour laquelle chaque élément $P_{i,j}^{(k)}$ représente la probabilité de passer de l'état $i$ à l'état $j$ au bout d'exactement $k$ étapes. Plus formellement :
\[
P_{i,j}^{(k)} = \mathbb{P}(X_k = j \mid X_0 = i).
% Je crois que les parenthèses autour du "n" ne sont pas nécessaire, tu ferais peut être mieux de les retirer -corentin
% Elles sont dans toutes les démos, donc je les aies laissées, je ne pense pas qu'elles dérangent ou ne soient pas correctes, si ? - Tom
\]

Cette matrice $P^k$ est obtenue en élevant la matrice de transition $P$ à la puissance $k$, ce qui revient à effectuer $k-1$ produits matriciels successifs :
\[
P^k = \underbrace{P \cdot P \cdot \ldots \cdot P}_\text{(k fois)}.
\]

Nous allons démontrer ce résultat en utilisant la propriété de markov ainsi que le produit matriciel.
La relation ci-dessus repose sur la propriété fondamentale des chaînes de Markov ; pour chaque étape, la probabilité de transition d'un état $i$ à un état $j$ dépend uniquement du dernier pas réalisé, et non de tous les précédents. En d'autres termes :
\[
\mathbb{P}(X_{k+1} = j \mid X_0 = i) = \sum_{n=1}^N \mathbb{P}(X_{k+1} = j \mid X_k = n) \cdot \mathbb{P}(X_k = n \mid X_0 = i).
\]

En termes matriciels, cette équation ce traduit donc par :
\[
P_{i,j}^{(k+1)} = \sum_{n=1}^N P_{i,n}^{(k)} \cdot P_{n,j}.
\]
Cela correspond précisément à la règle du produit matriciel, ce qui montre en conséquence que $P^{n+1} = P \cdot P^n$. Par récurrence, on en déduit donc que :
\[
P^k = \underbrace{P \cdot P \cdot \ldots \cdot P}_\text{(k fois)}.
\]
\end{tcolorbox}

\subsection{Classes de communication et d'équivalence}

\subsubsection{Classification des états}
\label{subsubsec: Les différents types d'états}

Dans une chaîne de Markov, chaque état peut être classifié en différentes catégories en fonction de ses laisons avec les autres états. Ces différentes classifications nous permettentrons d'analyser le comportement de la chaîne en temps long.

\begin{tcolorbox}[colback=white,colframe=blue!80!black,title=Accessibilité]
Soient $i$ et $j$ deux éléments de $E$. On dit que $j$ est \textbf{accessible} à partir de $i$ si :
\[
\exists n \geq 1, \quad P_{i,j}^{(n)} > 0.
\]
ce que l'on note $i \to j$. \\
Et si $i \to j$ et $j \to i$, alors les éléments $i$ et $j$ \textbf{communiquent}, on écrit :
\[
i \leftrightarrow j
\]
\end{tcolorbox}

En particulier, $i \to j$ si et seulement s'il existe une suite d'états $(i_0, i_1, \dots, i_n)$ avec $i_0 = i$ et $i_n = j$ telle que :
\[
P(X_n = j, \dots, X_1 = i_1 \mid X_0 = i) = P(i_{n-1}, j) \cdots P(i, i_1) > 0.
\]
Ce qui veut dire qu'il existe un chemin de probabilité strictement positive
qui mène de i à j.

\begin{tcolorbox}[colback=white,colframe=blue!80!black,title=État récurrent]
Un état $i$ est dit \textbf{récurrent} si, en partant de cet état, ill est certain d'y retourner.  
Autrement dit, en effectuant un nombre infini de transitions, on retourne à $i$ une infinité de fois.  

Plus rigoureusement, cela signifie que :  
\[
\mathbb{P}_i(\text{Num}_i(X) = \infty) = 1,
\]
où  
\[
\text{Num}_i(X) = |\{n \geq 0 \mid X_n = i\}|
\]
représente le nombre total de visites de l'état $i$, y compris la première.  

Un état qui n'est pas récurrent est dit \textbf{transient} ou \textbf{transitoire}.  
\end{tcolorbox}

\begin{tcolorbox}[colback=white,colframe=blue!80!black,title=État absorbant]
Un état $i$ est appelé \textbf{absorbant} si, une fois atteint, la chaîne reste dans cet état avec probabilité 1. Formellement, cet état satisfait :
\[
P_{i,i} = 1 \quad \text{et} \quad \forall j \neq i, \quad P_{i,j} = 0.
\]
A l'inverse, tout état ne respectant pas cette condition est appelé état \textbf{non-absorbant} ou encore état \textbf{évanescent}. \\
\end{tcolorbox}

\begin{tcolorbox}[colback=white,colframe=blue!80!black,title=État périodique]
Un état $i$ est dit \textbf{périodique} si la chaîne ne peut revenir à cet état qu'après un nombre d'étapes multiple (un certain entier $d > 1$), appelé la période de $i$. La période $d(i)$ est définie comme :
\[
d(i) = \text{PGCD}\{n \geq 1 \mid P_{i,i}^{(n)} > 0\}.
\]
Si $d(i) = 1$, l'état $i$ est dit \textbf{apériodique}, ce qui signifie qu'il est possible de revenir à cet état à chaque nouvelle transition /  sans contrainte de périodicité. \\
\end{tcolorbox}

% Est-ce qu'il faudrait ajouter le dernier type d'état suivant ? Il est super pratique dans notre cas mais ces notions ne sont pas encore introduites à ce niveau... - Tom
% Un dernier type d'état que nous étudierons est le type ergotique.
% Un état $i$ est dit \textbf{ergodique} si, en plus d'être récurrent, il est apériodique et appartient à une classe de communication contenant tous les autres états de la chaîne.
% Les états ergodiques permettent à la chaîne de converger vers une distribution stationnaire.
% TODO A AJOUTER APRES

\subsubsection{Classes de communication}
% REMPLACER LES ALPHA ET BETA PAR DES I ET J - Tom

\begin{tcolorbox}[colback=white,colframe=yellow!80!black,title=Backup à modifier/supprimer]
Soit $\alpha$, $\beta \in E$ et $i$, $j$ leurs indices respectifs dans $E$. On note $\alpha \rightarrow \beta$ si, et seulement si :
\[
\exists n \geq 0, \quad P_{i,j}^{(n)} > 0.
\]

On dit que les deux états $\alpha$ et $\beta$ communiquent si $\alpha \rightarrow \beta$ et $\beta \rightarrow \alpha$. On note alors $\alpha \leftrightarrow \beta$. Nous avons ainsi :
\[
\alpha \leftrightarrow \beta \quad \iff \quad \exists n \geq 0, \quad P_{i,j}^{(n)} > 0 \quad \text{et} \quad \exists m \geq 0, \quad P_{j,i}^{(m)} > 0.
\]
\end{tcolorbox}

Soit deux états communiquant $i$ et $j$. Selon la définition précédente, $\leftrightarrow$ est une relation d'équivalence dont les classes sont appelées \textbf{classes de communications} de la chaîne de Markov.
Autrement dit, dans une chaîne de Markov, une classe de communication est un ensemble d'états qui sont accessibles les uns depuis les autres. \\

L'ensemble des classes de communication, engendrées par $\leftrightarrow$, d'une chaîne de Markov sont \textbf{disjointes} et forment une \textbf{partition} de l'ensemble des états de cette chaîne.
Autrement dit, la relation d'équivalence $\leftrightarrow$ partitionne l'ensemble des états d'une chaîne de Markov en classes de communication disjointes. \\ % Tony, j'imagine que je ne m'en sortirais pas sans peuve ici, hein ? - Tom

Il est important de noter qu'une classe de communication appartienent systématiquement à l'une des deux catégories suivantes :
\begin{itemize}
\item \textbf{Fermée} : Une fois qu'un état de cette classe est atteint, il est impossible de quitter cette classe. Formellement, pour une classe $C$, cela signifie que :
  \[
  \forall i \in C, \quad \forall j \notin C, \quad \forall n \geq 1, \quad P_{i,j}^{(n)} = 0.
  \]
\item \textbf{Ouverte} : L'opposée de fermée ; il existe au moins un état de la classe qui permet de sortir de celle-ci. Plus formellement, cela signifie que :
  \[
  \exists i \in C, \quad \exists j \notin C, \quad \exists n \geq 1, \quad P_{i,j}^{(n)} > 0.
  \]
\end{itemize}

% est-ce que je devrais ajouter la phrase suivante ? Elle me semble bien pour introduire la notion de distribution stationnaire mais je pense qu'elle serait encore mieux dans la partie de définition des distributions stationnaires. - Tom
% Les classes de communication fermées jouent un rôle central dans l'étude asymptotique des chaînes de Markov, car toute chaîne converge vers une distribution stationnaire si et seulement si elle se trouve dans une classe fermée récurrente et apériodique.

\subsubsection{Classes d'équivalence}

Les classes d'équivalence sont des sous-ensembles d'états dans une chaîne de Markov qui partagent des propriétés spécifiques en plus d'être dans une même classe de communication. Une classe d'équivalence est donc une classe de communication qui regroupe des états voisins présentant une ou plusieurs des propriétés suivantes :

\begin{enumerate}
\item \textbf{Récurrence} : Tous les états de la classe sont récurrents. Cela signifie qu'une fois entré dans cette classe, il est certain que la chaîne revisitera un état de cette classe une infinité de fois au cours des transitions suivantes. Cette classe d'équivalence est systématiquement fermée.
\item \textbf{Transcience (transitivité)} : Dans certaines classes de communication, les états peuvent être transitoires, ce qui signifie qu'il est possible de quitter définitivement la classe par ces états. Cette classe d'équivalence, à l'opposé de la promière est toujours ouverte.
\item \textbf{Absorption} : Une classe d'équivalence récurrence (et fermée) peut également être absorbante. Cela signifie que, si la chaîne entre dans cette classe, elle ne pourra plus jamais en sortir.
\item \textbf{Apériodicité} : Tous les états d'une classe peuvent également être apériodiques, ce qui garantit l'absence de contraintes sur le moment où la chaîne peut revisiter un état donné. La classe d'équivalence ainsi formée peut être puverte ou fermée. \\
\end{enumerate}

Ainsi, une classe d'équivalence est une classe de communication fermée qui regroupe des états partageant une même propriété structurelle (tous les états de cette classe d'équivalence possède le même type d'état), telle que la récurrence, l'absorption, ou l'apériodicité. \\

Les classes d'équivalence jouent en conséquence un rôle crucial dans l'étude des chaînes de Markov. En effet, elles permettent de décomposer la chaîne en sous-ensembles analytiquement indépendants.

Par exemple ; les propriétés asymptotiques d'une chaîne dépendent fortement des classes d'équivalence fermées et récurrentes. % Je peux balancer cet "exemple" sans le justifier ? J'ai vraiment l'impression de tricher là. - Tom

\subsubsection{Exemples}

TODO \\
% J'ai eu un problème de génération de graphe par rapport à mes exemples, je les rajouterais de nouveau plus tard, quand j'aurai réglé ce problème. - Tom


%TODO \\
% A partager avec le point précédent :
% - [?] Convergences des chaines de Markov
% - [X] Loi stationnaire/invariante
% - ...

\subsection{Loi stationnaire}

Soit une chaîne de Markov $(X_n)_{n \in \mathbb{N}}$ à espace d'état $E$, et ayant pour matrice de transition $P$. \\

Une mesure $\pi = (\pi_i)_{i \in E}$ sur l'espace d'état $E$ est dite stationnaire si elle satisfait la relation $\pi = \pi P$. \\

Cela signifie que $\pi$ est un vecteur propre à droite de la transposée $P^{\top}$, associé à la valeur propre $1$ :
\[
\pi = \pi P \iff \pi^{\top} = P^{\top} \pi^{\top}.
\]

De plus, pour que $\pi$ soit une mesure de probabilité, elle doit respecter les conditions suivantes :
\[
\forall i \in E, \quad \pi_i \geq 0 \quad \text{et} \quad \sum_{i \in E} \pi_i = 1.
\]

\subsection{Comportement en temps long}
% REMPLACER LES ALPHA ET BETA PAR DES I ET J (si cette partie survie) - Tom

Une chaîne de Markov est dite \textbf{irréductible} si le graphe qui lui est associé est fortement connexe. % Corentin, tu signifies quoi par "fortement" ? - Tom
Autrement dit, tous les couples $(\alpha, \beta)$ de sommets du graphe, avec $\alpha \neq \beta$, communiquent mutuellement ; $\alpha \leftrightarrow \beta$. \\

Une chaîne de Markov est dite \textbf{récurrente} si tous ses états sont récurrents (voir sous-section ~\ref{subsubsec: Les différents types d'états}). \\ % A quoi sert cette définition ? Elle me semble un peu perdue. - Tom

Soit une chaîne de Markov $(X_n)_{n \in \mathbb{N}}$ à $\mathbb{N}$ états contenus dans $E$ et une matrice de transition $P$ associée à cette chaîne. Un état $\alpha$ est alors dit \textbf{récurrent positif} s'il existe une mesure stationnaire $\pi = (\pi_i)_{i \in E}$ telle que $\pi_\alpha > 0$.

Cela signifie donc qu'en moyenne le temps entre deux visites consécutives de l'état $\alpha$ est fini. \\

Une chaîne de Markov est dite \textbf{récurrente positive} si tous ses états sont récurrents positifs. \\

Si une chaîne de Markov est irréductible et récurrente positive, alors sa matrice de transition $P$, élevée à une puissance $k$, converge lorsque $k \to +\infty$. Plus formellement :
\[
L = \lim_{k \to +\infty} P^k
\]
Où la matrice limite $L$ est bien définie, et chaque ligne de cette matrice est égale à l'unique distribution stationnaire $\pi$ associée à la chaîne.

\newpage
\section{Fonctionnement de l'Algorithme de Hasting-Metropolis}

\subsection{Qu'est-ce qu'une méthode dite MCMC ?}

Les méthodes MC dites de Monte-Carlo ont pour but d'approcher de manière empirique des valeurs par un processus aléatoire répété un nombre suffisant de fois. Elles interviennent quand une résolution déterministe est trop difficile à obtenir, comme pour calculer des intégrales ou générer des échantillon de distributions statistiques. Les méthodes MC sont très fiables malgré l'aléatoire utilisé, pourvu que l'algorithme ait les bonnes optimisations telles que le nombre de répétitions, la précision exigée, la vitesse de convergence vers les valeurs cherchée, etc. \\
Les méthodes MC sont utilisées dans beaucoup de domaines des sciences (comme la physique, la chimie, la biologie, les mathématiques statistiques, l'intelligence artificielle, la finance et la cryptographie). \\

Les méthodes MCMC (Monte-Carlo par chaînes de Markov) sont une sous-famille des algorithmes MC qui construisent, dans leur fonctionnement, une chaîne de Markov dont la distribution stationnaire est celle que l’on souhaite estimer. \\
Il devient ainsi possible d'exploiter les propriétés des chaînes de Markov, notamment leur convergence vers une distribution stationnaire et leur propriété d'ergodicité, qui peuvent être plus utiles que la loi des grands nombres seule, utilisée dans les méthodes MC classiques. \\
La principale différence entre les méthodes MC et MCMC réside dans la dépendance entre les points générés. Les méthodes MC choisissent aléatoirement des points de l'espace, réalisant ainsi un échantillonnage direct de la distribution cible, avec des échantillons indépendants les uns des autres. Leur seul point commun est d'appartenir à l'intervalle des valeurs possibles de la distribution de départ. En revanche, les méthodes MCMC reposent sur une marche aléatoire : elles génèrent chaque nouveau point à partir du précédent selon une règle de transition déterminée. Cela crée une chaîne de points corrélés, qui possède les propriétés des chaînes de Markov. \\
Ainsi, lorsque la distribution est concentrée sur une certaine partie de l'espace, l’utilisation de MCMC sera généralement préférable, car elle permet d'obtenir, dans un temps raisonnable, des échantillons représentatifs de cette distribution. \\

On pourra citer les méthodes MCMC d'échantillonnage de Gibbs (et ses variantes comme le Blocked Gibbs Sampling ou l'Adaptive Gibbs Sampling), qui prennent leur intérêt lorsque les variables de la distribution sont conditionnelles. Il existe aussi l’échantillonnage par tranches, qui échantillonne progressivement la distribution par "tranches" de valeurs, ou bien des optimisations de HM, telles que la méthode hamiltonienne de Monte-Carlo, qui calcule le gradient pour faciliter la marche du programme et rejeter moins de valeurs, tout en gagnant en performances. \\

\subsection{Definitions}

TODO \\
% Pour les defs je pensais par exemples à celles :
% - De l'échantillonnage (en maths)
% - Espace dense de probabilité (je crois c'est ça), c'est le P(x) qu'on retrouve dans les démos
% - ...

\subsection{Description des étapes de l'algorithme}

Voici une description de cet algorithme :

\begin{enumerate}
\item Il faut avant tout choisir un point $x_0$, comme étant le premier échantillon de notre loi cible, et aussi une probabilité de transition $g$, en donnant les fonctions correspondant à $g_y(x)$ et $g(x)$, $\forall x,y \in E$.  

\item Ensuite nous itérons sur $t$ allant de 0 à $N$ (notre nombre d'itérations voulu)  \begin{itemize}
    \item On tire $x$ avec $g_{x_t}(x)$  
    \item On pose $\alpha \coloneqq \frac{ \pi(x) g_x(x_t)}{\pi(x_t) g_{x_t}(x)}$ (notons que si nous ne possédons qu'une densité proportionnelle à $\pi$, $f$ nous pouvons poser $\alpha \coloneqq \frac{ f(x) g_x(x_t)}{f(x_t) g_{x_t}(x)}$)  
    \item On tire $u \in [0;1]$, tel que $u  \sim \mathcal{U}(0,1)$  \begin{itemize}
        \item Si $u \leqslant \alpha$ alors : $x_{t+1} \coloneqq x$  
        \item Sinon : on conserve l'état précédent $x_{t+1} \coloneqq x$  
        \end{itemize}
    \end{itemize}
\item La séquence $\{x_0,x_1, ... ,x_{N-1}\}$ constitue donc l'échantillon obtenue à partir de la chaîne de Markov associé à la loi $\pi$
\end{enumerate}

\subsection{Preuve de l'algorithme}

TODO \\
% Démontrer pourquoi l'algorithme marche en se basant sur la catégorie des chaines de Markov pour appuyer la preuve.

\subsection{Exemple d'application}

\begin{tikzpicture}
    % Axes
    \draw[->,black,thick] (-1,0) -- (7,0) node[right] {$x$}; % Axe des abscisses en noir
    \draw[->,thick] (0,-0.2) -- (0,2.5) node[above] {$\pi(x)$}; % Axe des ordonnées

    % Distribution cible (exemple : distribution normale)
    \begin{scope}
        \clip (-0.5,-0.2) rectangle (6.5,2);
        \draw[thick,gray!50] plot[domain=0:6, samples=100] (\x, {2*exp(-(\x-3)^2)});
    \end{scope}
    \node at (6,1.8) {Distribution cible $\pi(x)$};

    % Points de la chaîne de Markov sur l'axe x
    \filldraw[blue] (1.5,0) circle (1pt) node[below] {$x_0$}; % Encore légèrement déplacé vers la droite
    \filldraw[blue] (2,0) circle (1pt) node[below] {$x_1$};
    \filldraw[blue] (3.2,0) circle (1pt) node[below] {$x_2$};
    \filldraw[blue] (4,0) circle (1pt) node[below] {$x_3$}; % x3 inchangé
    \filldraw[red] (2.5,0) circle (1pt) node[below=8pt] {$x'_{}$ rejeté}; % Aligné avec x'' rejeté
    \filldraw[red] (4.4,0) circle (1pt) node[below=8pt] {$x''$ rejeté}; % Déplacé légèrement plus à gauche

    % Points correspondants sur la courbe
    \foreach \x in {1.5,2,3.2,4} {
        \filldraw[blue] (\x,{2*exp(-(\x-3)^2)}) circle (1pt);
        \draw[dashed,gray] (\x,0) -- (\x,{2*exp(-(\x-3)^2)});
    }
    % Points rejetés sur la courbe
    \filldraw[red] (2.5,{2*exp(-(2.5-3)^2)}) circle (1pt);
    \draw[dashed,gray] (2.5,0) -- (2.5,{2*exp(-(2.5-3)^2)});
    \filldraw[red] (4.4,{2*exp(-(4.4-3)^2)}) circle (1pt);
    \draw[dashed,gray] (4.4,0) -- (4.4,{2*exp(-(4.4-3)^2)});

    % Transitions acceptées (flèches bleues bien espacées)
    \draw[->,blue,thick] (1.5,0) .. controls (1.8,0.3) .. (2,0);
    \draw[->,blue,thick] (2,0) .. controls (2.6,0.3) .. (3.2,0);
    \draw[->,blue,thick] (3.2,0) .. controls (3.6,0.5) .. (4,0); % Flèche vers x3

    % Transitions rejetées (flèches rouges en pointillé repositionnées)
    \draw[->,red,dashed,thick] (3.2,0) .. controls (2.8,0.8) .. (2.5,0); % x2 vers x' rejeté
    \draw[->,red,dashed,thick] (4,0) .. controls (4.2,0.6) .. (4.4,0); % Flèche de x'' moins haute

    % Légende
    \draw[blue,thick] (1,-1.2) -- (1.5,-1.2) node[right] {Transitions acceptées};
    \draw[red,dashed,thick] (1,-1.6) -- (1.5,-1.6) node[right] {Transitions rejetées};
    \filldraw[blue] (1,-2) circle (1pt) node[right] {Échantillons acceptés};
    \filldraw[red] (1,-2.4) circle (1pt) node[right] {Échantillons rejetés};

\end{tikzpicture}



\begin{tikzpicture}
    % Titre du diagramme
    \node[anchor=south] at (2,1.5) {\textbf{Diagramme initial à optimiser}};

    % Nœuds du diagramme
    \node (A) at (0,0) {A};
    \node (B) at (2,1) {B};
    \node (C) at (4,0) {C};
    \node (E) at (3,-2) {E};
    \node (D) at (1,-2) {D};
    
    % Trajets
    \draw[thick] (A) -- (C) -- (D) -- (E) -- (B) -- (A);
\end{tikzpicture}

\begin{tikzpicture}
    \node[anchor=south] at (2,1.5) {\textbf{Exemple de proposition retenue par HM}};

    \node (A) at (0,0) {A};
    \node (B) at (2,1) {B};
    \node (C) at (4,0) {C};
    \node (E) at (3,-2) {E};
    \node (D) at (1,-2) {D};

    \draw[thick] (A) -- (B); 
    \draw[thick] (E) -- (D) -- (C); 
    \draw[green, thick, dashed] (A) -- (C);
    \draw[green, thick, dashed] (B) -- (E);
    \draw[blue, thick] (B) -- (C);
    \draw[blue, thick] (A) -- (E);
\end{tikzpicture}

\begin{tikzpicture}
    \node[anchor=south] at (2,1.5) {\textbf{Exemple de proposition rejetée par HM}};

    \node (A) at (0,0) {A};
    \node (B) at (2,1) {B};
    \node (C) at (4,0) {C};
    \node (E) at (3,-2) {E};
    \node (D) at (1,-2) {D};

    \draw[thick] (A) -- (C) -- (D); 
    \draw[thick] (B) -- (E); 
    \draw[orange, thick, dashed] (A) -- (B);
    \draw[orange, thick, dashed] (D) -- (E);
    \draw[red, thick] (D) -- (B);
    \draw[red, thick] (A) -- (E);
\end{tikzpicture}










% On va sûrement modéliser HM avec Godot (si on est chauds) ou en Python 3 (on peut refaire l'algo de cette vidéo où les sources sont en description \url{https://papyrus.bib.umontreal.ca/xmlui/bitstream/handle/1866/6231/Mireuta_Matei_2011_memoire.pdf}).
% L'objectif ici est de parler de NOTRE implémentation surtout je pense, ça sera la majeure partie de l'oral.

\subsection{Comparaison avec d'autres méthodes d'échantillonnage}

TODO \\
% Je sais pas trop où placer cette partie, voire peut-être l'enlever mais je trouve intéressant et pertinent de comparer les méthodes déchantillonnage, peut-être le placer dans la parties des méthodes MCMC si besoin. Pour moi la logique voudrait qu'on aie cette partie après avoir montré notre algo mais à voir.

\newpage
\section{Conclusion}

TODO \\

\newpage
\section{Sources}

Documents textuels :
\begin{itemize}
\item \url{https://fr.wikipedia.org/wiki/Chaîne_de_Markov}
\item \url{https://dms.umontreal.ca/~bedard/BergeronL_rapport_final.pdf}
\item \url{https://www.math.univ-paris13.fr/~tournier/fichiers/agreg/2014/cours_markov.pdf}
\item \url{https://www.math.u-bordeaux.fr/~mchabano/Agreg/ProbaAgreg1213-COURS5-CM.pdf}
\item \url{https://www.imo.universite-paris-saclay.fr/~pierre-loic.meliot/agreg/markov.pdf}
\item \url{https://www.college-de-france.fr/fr/agenda/cours/apprentissage-et-generation-par-echantillonnage-aleatoire/algorithme-de-metropolis-hasting}
\item \url{https://www.math.u-bordeaux.fr/~mibonnef/mimse-markov/recurence-transience.pdf}
\item \url{https://fr.wikipedia.org/wiki/Processus_stochastique}
\item \url{https://fr.wikipedia.org/wiki/Méthode_de_Monte-Carlo_par_chaînes_de_Markov}
\item \url{https://www.math.u-bordeaux.fr/~mchabano/Agreg/ProbaAgreg1314-COURS5-CM.pdf} \\
\end{itemize}

Documents vidéos :
\begin{itemize}
\item \url{https://www.youtube.com/watch?v=yCv2N7wGDCw}
\item \url{https://www.youtube.com/watch?v=MxI78mpq_44}
\item \url{https://www.youtube.com/watch?v=e0ZHDK4DSEI&list=PLWoShwK0FEjovcc32x9LbpDTf8pquPimV}
\item \url{https://www.youtube.com/playlist?list=PLWoShwK0FEjovcc32x9LbpDTf8pquPimV} \\
\end{itemize}

\end{document}
